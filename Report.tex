\documentclass{article}
\usepackage{graphicx}
\begin{document}
\title{Speech Command Recognition}
\date{\today}
\maketitle
\begin{itemize}


\item{\textbf{OVERVIEW:}}\\

Speech recognition is the process of converting\vspace{3} human sound signals into words \vspace{3} or instructions. It is based on speech. It is an important research direction of speech signal processing\vspace{3} and a branch of pattern recognition.\vspace{3}Speech recognition applications include voice user interfaces such as voice dialing (e.g. ”call home”), call routing (e.g. ”I would like to make\vspace{3} a collect call”), domestic appliance control, search key words\vspace{3} (e.g. find a podcast where particular words were spoken), simple data entry\vspace{3} (e.g., entering a credit card number).\vspace{3} In similar way Speech voice recognition model is based on concepts of \vspace{3}Convolution, LSTM , Attention and recognise pretrained voice with accuracy of \textbf{99.6\%}..\\

\item{\textbf{DATA:}}\\

1: Set 16KHz as sampling rate.\\

2: Record 80 utterances of each command.\\

3: Save samples of each command in different folders.\\


\item[*]{Data/forward}.

\item[*]{Data/back}.

\item[*]{Data/left}.

\item[*]{Data/right}.

\item[*]{Data/stop}.\\
\newpage

\item{\textbf{Description:}}\\

1: Using Audacity software we have generated the data set i.e voice commands Forward, Back, Left, Right, Stop with 80 utterances for each.\\

2: We have splitted the data into two categories i.e Train and Test. where 20\% of the data set are used for testing and the rest for training.The shape of the data set which is used for testing will be \textbf{(1775, 49, 39, 1)} and for training set, shape will be \textbf{(7033, 49, 39, 1)}\\

3: We have also used Melspectogram for extracting features from the data set and also normalization is used so that the model will achieve the converge point point within few epochs.\\

4: LSTM neural network is also used so that our weights will get updated without showing any \textbf{vanishing gradient problem} during training of models. \\

5: Attention is used to get the required data even from a complex sentence, Also batch normalization is used in order to prevent unexpected behaviour of the weights.\\

6: Finally an audio is recorded from user and the model will predict this data based on the already given training and the desired result will be displayed on screen.\\


\item{\textbf{The model is successfully built and has achieved the highest accuracy of 99.6\%}}\\

\newpage

\item{\textbf{Model Summary}}\\

\begin{figure}[h]
\includegraphics[scale=0.5]{qwer.png}
\end{figure}

\\

\item{\textbf{RUN:}}\\

The Code is written using Google Colab:\\

\item[1.]{Open ColabNotebook.ipynb and change Runtime to GPU.}

\item[2.]{Upload Speech-Recognition/Speech to Colab.}


\item[3.]{Change data-dir in all cells to point to Speech-Recognition/speech.}

\item[4.]{Run the cells in the same order in Notebook Test.}
\\

\item{\textbf{TEST:}}\\


1: Locate the folder where you save your model.h5 file.\\

2: Start speaking when you see mike in the bottom right pane of the task bar or see red blinking dot in the title bar.\\

\\


\item{\textbf{Language Used:}}\\

PYTHON\\

\item{\textbf{Libraries and Packages Used:}}\\

KAPRE, SCIKIT LEARN, SOUND FILE, TENSORFLOW.\\

\end{itemize}
\end{document}